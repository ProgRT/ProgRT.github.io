\documentclass[aspectratio=169]{beamer}
\usepackage[french]{babel}
\usepackage[
	backend=biber,
	style=numeric,
	url=false,
	isbn=false
]{biblatex}


%%%%%%%%%%%%%%
% Librairies %
%%%%%%%%%%%%%%

\usepackage{hyperref, nicematrix, rsymbols, booktabs, tikz, qrcode}
\usepackage[subpreambles=true, mode=build]{standalone}
\usetikzlibrary{math}

%%%%%%%%%
% Style %
%%%%%%%%%

\usetheme[progress, sectiontoc]{montreal}
\usecolortheme{pavlov}
\setbeamertemplate{title page}[nologo compact][left]

\renewcommand{\arraystretch}{1.8}
\renewcommand*{\bibfont}{\footnotesize}

%%%%%%%%%%%%%
% Utilitées %
%%%%%%%%%%%%%

\usepackage{graphicx}
\newcommand{\icone}[1]{%
    \raisebox{-1ex}{%
        \includegraphics[height=4ex]{_icones/#1}
    }
}

\newcommand{\annotatefactor}[2]{
    \begin{tikzpicture}[
        annotation/.style={
            red,
            thick,
            ->,
            shorten >=3pt,
            shorten <=3pt
        }
    ]

    \draw [annotation]
    (2-#1.east)
    to[bend left=45] node[right] {$\times #2$}
    (4-#1.east)
    ;

    \end{tikzpicture}
}

\newcommand{\annotatedelta}[2]{
    \begin{tikzpicture}[
        annotation/.style={
            red,
            thick,
            ->,
            shorten >=3pt,
            shorten <=3pt
        }
    ]

    \draw [annotation]
    (#1-2.east)
    to node[below] {$\times #2$}
    (#1-3.west)
    ;

    \end{tikzpicture}
}


\def\vs{{\em vs}}


%%%%%%%%%%%%%%%%%%%%%%%%%%%%%%%%%
% Meta-information. Kind of ...
%%%%%%%%%%%%%%%%%%%%%%%%%%%%%%%%%

\title{Apprivoiser le mode PAV+}
%\author{Nicolas Blais St-Laurent, {\small inh}}
\institute{\nchum}
\titlebg{img/Pavlov}
\addbibresource{./references.bib}

%%%%%%%%%%%%%%%%
\begin{document}
%%%%%%%%%%%%%%%%

\maketitle

%%%%%%%%%%%%%%%%%%%%%%
\section{Généralités}
%%%%%%%%%%%%%%%%%%%%%%

\begin{frame}{Classification des modes spontanés}
    \centering
    \includestandalone{tbl/tbl-classification}
\end{frame}

\begin{frame}{Boucle de rétroaction du contrôle respiratoire}
    \centering
    \includestandalone{fig/fig-graph}
\end{frame}


\begin{frame}{Le mode PAV+}
    \begin{block}{Équation du mouvent de l'air}
        $$\Delta P_{tot} = \Delta V_{(L)} \cdot E_{(\frac{cmH_2O}{L})} + \dot{V}_{(L/s)} \cdot R_{(\frac{cmH_2O}{L/s})}$$
    \end{block}

    \begin{block}{Calcul de l'assistance}
        $$ \Delta P_{vent} = \Delta P_{tot} \cdot \%_{support} $$
    \end{block}
\end{frame}


\begin{frame}{PAV+ \vs\ VS-AI}
    \renewcommand{\arraystretch}{1.2}

    \begin{block}{VS-AI (AI = 10 \cmh)}

        \begin{NiceTabular}{l c c}
        & Commande élevée & Commande faible \\ 
        \hline
        \Pmus & 10 & 5 \\
        \Pvent & 10 & 10 \\
        \Ptot & 20 & 15 \\
        \hline
        \CodeAfter
        \uncover<2>{
            \annotatefactor{2}{2}
            \annotatefactor{3}{3}
        }
        \uncover<3->{
            \annotatedelta{2}{0,5}
            \annotatedelta{4}{0,75}
        }
        \end{NiceTabular}
    \end{block}

    \begin{block}{PAV+ (\Psupport = 50~\%)}

        \begin{NiceTabular}{l c c}
        & Commande élevée & Commande faible \\ 
        \hline
        \Pmus & 10 & 5 \\
        \Pvent & 10 & 5 \\
        \Ptot & 20 & 10 \\
        \hline

        \CodeAfter
        \uncover<4>{
            \annotatefactor{2}{2}
            \annotatefactor{3}{2}
        }
        \uncover<5->{
            \annotatedelta{2}{0,5}
            \annotatedelta{4}{0,5}
        }
        \end{NiceTabular}
    \end{block}
\end{frame}

%%%%%%%%%%%%%%%%%%%%%%%%
\section{Mise en route}
%%%%%%%%%%%%%%%%%%%%%%%%

\begin{frame}{Travailler avec le PB-980}
    \begin{columns}[c]
        \begin{column}{0.5\textwidth}
            \begin{itemize}
                \item Pas de bouchons
                \item Tout brancher
                \item Baisser le volume
                \item Faire un ATR
                \item Programmer à l'arrivée
                \item Les courbes et valeurs numériques peuvent être changées
            \end{itemize}
        \end{column}

        \begin{column}{0.5\textwidth}
            \centering
            \includegraphics[height=0.8\textheight]{img/980}
        \end{column}
   \end{columns}
\end{frame}

\begin{frame}{Paramètres de départ}
    \centering
    \includestandalone[mode=tex]{tbl/tbl-parametres}
\end{frame}

%%%%%%%%%%%%%%%%%%%%%%%%%%%
\section{Gestion clinique}
%%%%%%%%%%%%%%%%%%%%%%%%%%%

\begin{frame}{Indicateurs de travail}
    \centering
    \includestandalone{tbl/tbl-indicateurs}
\end{frame}

\begin{frame}{Calculer la \Pmus}
    \begin{equation}
        \Delta P_{tot} = \frac{\Delta P_{vent}}{\%_{assist}}
    \end{equation}

    \begin{equation}
        \Delta P_{mus} = \frac{\Delta P_{vent}}{\%_{assist}} \cdot (1-\%_{assist}%)
    \end{equation}
\end{frame}


\begin{frame}{\Pmus\ 5 à 10 cmH₂O}
    \centering
    \includestandalone[mode=tex]{fig/graphpmus}
\end{frame}

\begin{frame}{\Pmus\ \vs\ \Pvent}
    \centering
    \resizebox{!}{0.83\textheight}{
        \includestandalone[mode=tex]{tbl/tbl-ampl-pgf}
    }
\end{frame}


\begin{frame}{Quand ça ne marche pas}
    \begin{itemize}
        \item Fuite
        \item Commande respiratoire
        \item Autodéclenchement
        \item Variabilité respiratoire
        \item Suramplification
    \end{itemize}
\end{frame}

\begin{frame}{\Psupport\ \vs\ amplification}
\tikzmath{
    for \assist in {.1, .2, .333333, .5, .666666, .75, .9, .95} {
        \ampl{\assist} = 1/(1-\assist);
    };
}

    \resizebox{\textwidth}{!}{
        \begin{tikzpicture}[]
            \def\factor{20};
            \draw [thick](1.3,.5) -- (\factor*.98,.5);
            \foreach \assist in {.1, .2, .333333, .5, .666666, .75, .9, .95} {
                \draw [thick] (\assist*\factor, .25) -- (\assist*\factor, .75);
                \node at(\assist * \factor, 0) {\pgfmathprintnumber[verbatim ,fixed, skip 0.]{\assist}};
                    \node at(\assist * \factor, 1) {x \pgfmathprintnumber[verbatim]{\ampl{\assist}}};
            };
        \end{tikzpicture}
    }
\end{frame}

%\begin{frame}{Merci pour votre attention}
%    Revoir cette présentation (et pous encore!): \qrcode[hyperlink]{https://progrt.github.io}
%\end{frame}

\begin{frame}[allowframebreaks]{Bibliographie}
   \printbibliography
\end{frame}

\end{document}
